\documentclass{article}

\usepackage[a4paper,margin=1.5cm]{geometry}
\usepackage{amsmath}
\usepackage{amssymb}
\usepackage{graphicx}
\usepackage{cancel}
\usepackage{mhchem}
\usepackage{hyperref}
\usepackage{fourier}
\usepackage[dvipsnames]{xcolor}\hypersetup{linkbordercolor=blue}

\title{Geodesic Equation}
%\author{Lei Huang}
%\date{\today}

\everymath{\displaystyle}

\begin{document}
\maketitle

\section{Geodesic equation}
\subsection{Geodesic Equation as Conventionally Defined}
$\ddot{p}_k + \Gamma_{ij}^k \dot{p}_i \dot{p}_j = 0 $ (Explain a bit ***)

\subsection{Geodesic Equation as Used Here}
Geodesic equation is a system of second-order autonomous ODEs in the parameter space. 

$p''(t) = f(p, p')$.

Let $v(t) \equiv p'(t)$.

\begin{equation}
\left    \{
                \begin{array}{ll}
                  p' = v \\
                  v' = f(p, v)
                \end{array}
              \right.
\end{equation}

If $a(t) \equiv p''(t) = 0$:
$A(t) \equiv f''(p(t)) = A_{\perp} + A_{\parallelslant}$ 

Since one way of interpreting geodesics is free motion of a point particle on a manifold, $a(t)$ needs to take a value that renders $A_{\parallelslant}=0$.

Since $A_{\parallelslant} = P_{\parallelslant} A = J (J^T J)^{-1} J^T A$, and any vector $u$ in $P$ is mapped to the prediction space as $J u$, it follows that $a = - (J^T J)^{-1} J^T A$ leads to $A_{\parallelslant}=0$.

\begin{equation}
\left    \{
                \begin{array}{ll}
                  p' = v \\
                  v' = - (J^T J)^{-1} J^T A
                \end{array}
              \right.
\end{equation}

$A$ is usually calculated using finite difference:

$A(t) \equiv f''(p(t)) \approx \frac{f'(p(t+\delta t)) - f'(p(t-\delta t))}{2 \delta t} \approx \frac{ \frac{f(p(t+2\delta t)) - f(p(t))}{2 \delta t} - \frac{f(p(t)) - f(p(t-2\delta t))}{2 \delta t}}{2 \delta t} = \frac{f(p(t+2\delta t)) + f(p(t-2\delta t)) - 2f(p(t))}{4 (\delta t)^2}$

Let $\Delta t = 2 \delta t$:

$A(t) \approx \frac{f(p(t+\Delta t)) + f(p(t-\Delta t)) - 2f(p(t))}{(\Delta t)^2} \approx \frac{f(p(t)+v(t)\Delta t)) + f(p(t)-v(t)\Delta t)) - 2f(p(t))}{(\Delta t)^2} $
\newline

New notes (\textcolor{OliveGreen}{2-24-16, Wed}): 

Let's think in this way. 

\begin{equation}
\left    \{
                \begin{array}{ll}
                  p' = v \\
                  v' = a
                \end{array}
              \right.
\end{equation}

We need to find $a$ that will render $A_{\parallelslant}=0$. 

Given $p, v, a$, $A = f''(t) \approx \frac{f(p(t+\Delta t)) + f(p(t-\Delta t)) - 2f(p)}{(\Delta t)^2}$, where

\begin{equation}
\left    \{
                \begin{array}{ll}

f(p(t+\Delta t)) \approx f \left(p + v \Delta t + \frac{1}{2}a (\Delta t)^2 \right) \approx f(p+v \Delta t) + \frac{1}{2} J a (\Delta t)^2 \\

f(p(t-\Delta t)) \approx f \left(p - v \Delta t + \frac{1}{2}a (\Delta t)^2 \right) \approx f(p-v \Delta t) + \frac{1}{2} J a (\Delta t)^2

                \end{array}
              \right.
\end{equation}

Therefore, $A = \frac{1}{(\Delta t)^2}\left( f(p+v \Delta t) + \frac{1}{2} J a (\Delta t)^2 + f(p-v \Delta t) + \frac{1}{2} J a (\Delta t)^2 - 2f(p) \right) = A_{fd} + J a$

$A_{\parallelslant} = P_{\parallelslant} A = \left(J (J^T J)^{-1} J^T\right) \left(A_{fd} + J a \right) = 0$, so $\boxed{a=-(J^T J)^{-1} J^T A_{fd}}$. 
\newline

Proof of geodesics follow isocurves when degeneracy is one.

Sketch of proof: Geodesics have constant speed on manifold; initial speed is zero, so $y$ doesn't move along geodesic motion; which is the definition of isocurves. 




\subsection{Different Time Parametrizations of Geodesic Equation}
The above choice of $a(t)$ leads to constant speed in prediction space, characteristic of free motion in that space. But it has the drawback that an integration of the equation corresponding to a motion towards any boundary of the manifold would run into singularity, and a geodesic motion parameterized in such a different way that the speed is constant in parameter space would avoid this problem: (emphasize time parametrization ***)

$a = a_{\parallelslant} + a_{\perp}$ (in the $v$ direction ***)

Constant speed in $P$ \\
$\Longrightarrow \frac{\Vert v \Vert}{dt} = 0 = a_{\parallelslant} = 0$ \\
$\Longrightarrow a \leftarrow a - a_{\parallelslant} = a - \frac{a \cdot v}{v \cdot v} v$

Argument: $p(t) \rightarrow p(\tau)$ does not change the trace (? ***) of the curve in $P$, hence geodesic motion in prediction space is preserved.

\section{Setup}
\subsection{Notation}
$\leftarrow, \rightarrow$
$P$ (? ***)     

\subsection{$f$}
$y=f(p)$

$p \in P$, parameter in parameter space

$y \in Y$, prediction in prediction space

\subsection{Linearization of $f$}
$J \equiv Df$, Jacobian

$P_{\parallelslant} =  J (J^T J)^{-1} J^T$, projection operator onto tangent space (explained... ***)

$P_{\perp} = I - P_{\parallelslant} = I - J (J^T J)^{-1} J^T$, projection operator onto normal subspace

$g \equiv J^T J$, metric tensor
When $g$ is (close to) singular, one conventional way of making it less so is $g \leftarrow g + \lambda I$. 

\end{document}